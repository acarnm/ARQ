\documentclass[aspectratio=169,11pt]{beamer}

% Tema y paquetes
\usetheme{Madrid}
\usecolortheme{seahorse}
\usefonttheme{professionalfonts}
\setbeamertemplate{navigation symbols}{}
\usepackage[spanish]{babel}
\usepackage[utf8]{inputenc}
\usepackage{graphicx}
\usepackage{hyperref}
\usepackage{listings}
\usepackage{xcolor}
\usepackage{tikz}

% Estilo de código
\lstset{
  basicstyle=\ttfamily\footnotesize,
  keywordstyle=\color{blue},
  stringstyle=\color{green!50!black},
  commentstyle=\color{gray},
  frame=single,
  breaklines=true,
  showstringspaces=false
}

% Datos de la presentación
\title{Introducción y uso práctico de Raspberry Pi}
\subtitle{Conceptos, modelos, usos y programación}
\author{Abel Carnicero\texorpdfstring{\\ \url{acarnm@unileon.es}}{}}
\date{\today}

\begin{document}

% Portada
\begin{frame}
  \titlepage
\end{frame}

% Índice
\begin{frame}{Contenido}
  \tableofcontents
\end{frame}

% ============================================================
\section{¿Qué es Raspberry Pi?}

\begin{frame}{Definición}
  \begin{itemize}
    \item Una \textbf{computadora de placa única} (SBC: Single Board Computer).
    \item Diseñada por la \textit{Raspberry Pi Foundation} (Reino Unido).
    \item Objetivo: \textbf{fomentar la enseñanza de la informática y electrónica}.
    \item Muy económica, versátil y de bajo consumo.
  \end{itemize}
\end{frame}

\begin{frame}{Partes principales}
  \begin{columns}
    \column{0.5\textwidth}
    \begin{itemize}
      \item CPU y GPU (procesador ARM)
      \item RAM
      \item Puertos USB y HDMI
      \item Conector GPIO
      \item Conector de cámara (CSI)
      \item Lector microSD
    \end{itemize}
    \column{0.5\textwidth}
    % Imagen de Raspberry Pi - reemplazar con ruta real
    % \includegraphics[width=\textwidth]{img/raspberry_pi.jpg}
    \begin{center}
      \textcolor{gray}{\textit{[Insertar imagen de Raspberry Pi]}}
    \end{center}
  \end{columns}
\end{frame}

\begin{frame}{Precauciones y buen uso}
  \begin{itemize}
    \item Manipularla en superficies \textbf{no conductoras}.
    \item Evitar tocar los pines GPIO con electricidad estática.
    \item Usar una fuente de alimentación adecuada (5V, 2.5A o más).
    \item Apagar correctamente el sistema operativo antes de desconectar la energía.
    \item Mantener una buena ventilación para evitar sobrecalentamiento.
  \end{itemize}
\end{frame}

% ============================================================
\section{Modelos y diferencias}

\begin{frame}{Modelos principales}
  \begin{itemize}
    \item Raspberry Pi 1 (A, B)
    \item Raspberry Pi 2
    \item Raspberry Pi 3 (con Wi-Fi y Bluetooth)
    \item Raspberry Pi 4 (más RAM, USB 3.0, dual HDMI)
    \item Raspberry Pi 5 (mejor GPU y rendimiento)
    \item Raspberry Pi Zero / Zero 2W (versión reducida)
  \end{itemize}
\end{frame}

\begin{frame}{Comparativa rápida}
  \begin{center}
    \begin{tabular}{lccc}
      \hline
      Modelo & CPU     & RAM    & Conectividad           \\
      \hline
      Pi 3B+ & 1.4 GHz & 1 GB   & Wi-Fi, BT 4.2          \\
      Pi 4B  & 1.5 GHz & 2-8 GB & Wi-Fi, BT 5.0, USB 3.0 \\
      Pi 5   & 2.4 GHz & 4-8 GB & Wi-Fi, BT 5.3, PCIe    \\
      \hline
    \end{tabular}
  \end{center}
\end{frame}

% ============================================================
\section{Qué se puede hacer con Raspberry Pi}

\begin{frame}{Usos comunes}
  \begin{itemize}
    \item Centro multimedia (Kodi, Plex)
    \item Servidor casero (web, archivos, impresión)
    \item Retroconsola (RetroPie)
    \item Estación meteorológica
    \item Domótica e IoT
    \item Proyectos de robótica y control
    \item Aprendizaje de programación
  \end{itemize}
\end{frame}

\begin{frame}{Proyectos educativos}
  \begin{itemize}
    \item Programación en Python y Scratch
    \item Uso de sensores y actuadores
    \item Proyectos con LEDs, servos, cámaras
    \item Introducción a Linux y redes
  \end{itemize}
\end{frame}

% ============================================================
\section{Instalación y configuración}

\begin{frame}{Sistema operativo}
  \begin{itemize}
    \item Usar \textbf{Raspberry Pi Imager} (oficial) o balenaEtcher.
    \item Sistema recomendado: \textbf{Raspberry Pi OS (basado en Debian)}.
    \item Alternativas: Ubuntu, RetroPie, LibreELEC, etc.
  \end{itemize}
\end{frame}

\begin{frame}{Pasos de instalación}
  \begin{enumerate}
    \item Descargar Raspberry Pi Imager desde \url{https://www.raspberrypi.com/software/}
    \item Seleccionar sistema operativo e imagen.
    \item Elegir la tarjeta SD.
    \item Configurar (Wi-Fi, usuario, SSH opcional).
    \item Grabar imagen y arrancar la Raspberry Pi.
  \end{enumerate}
\end{frame}

% ============================================================
\section{Uso básico y software}

\begin{frame}[fragile]{Primeros pasos}
  \begin{itemize}
    \item Primer arranque: configuración de idioma, red y actualización.
    \item Acceso remoto: SSH, VNC o escritorio remoto.
    \item Comandos básicos de Linux:
  \end{itemize}

  \begin{lstlisting}[language=bash]
sudo apt update && sudo apt upgrade
pwd      # Mostrar directorio actual
ls       # Listar archivos
cd       # Cambiar de directorio
python3  # Iniciar interprete de Python
  \end{lstlisting}
\end{frame}

\begin{frame}{Software recomendado}
  \begin{itemize}
    \item \textbf{VS Code} o \textbf{Thonny} para programar.
    \item \textbf{Python 3}: lenguaje principal de enseñanza.
    \item \textbf{GPIO Zero}: control sencillo de hardware.
    \item \textbf{Node-RED}: entorno visual para IoT.
    \item \textbf{Mosquitto}: servidor MQTT para comunicación IoT.
  \end{itemize}
\end{frame}

% ============================================================
\section{Programación y ejemplos}

\begin{frame}[fragile]{Ejemplo básico en Python}
  \begin{lstlisting}[language=Python]
from gpiozero import LED
from time import sleep

led = LED(17)

while True:
    led.on()
    sleep(1)
    led.off()
    sleep(1)
  \end{lstlisting}
  \textbf{Ejemplo:} parpadeo de LED conectado al pin GPIO17.
\end{frame}

\begin{frame}[fragile]{Lectura de sensor}
  \begin{lstlisting}[language=Python]
from gpiozero import MotionSensor

pir = MotionSensor(4)

while True:
    pir.wait_for_motion()
    print("Movimiento detectado!")
    pir.wait_for_no_motion()
  \end{lstlisting}
\end{frame}

% ============================================================
\section{Cierre y recursos}

\begin{frame}{Consejos finales}
  \begin{itemize}
    \item Practica con proyectos sencillos antes de complicar.
    \item Documenta tus pasos y errores.
    \item Participa en la comunidad (foros, GitHub, Reddit).
  \end{itemize}
\end{frame}

\begin{frame}{Recursos recomendados}
  \begin{itemize}
    \item \url{https://www.raspberrypi.org/learn/}
    \item \url{https://projects.raspberrypi.org/}
    \item Libros: \textit{“Adventures in Raspberry Pi”}, \textit{“Exploring Raspberry Pi”}.
    \item YouTube: canales de Domótica, Electrónica y Linux.
  \end{itemize}
\end{frame}

\begin{frame}[plain]
  \centering
  \Huge ¡Gracias!
\end{frame}

\end{document}
