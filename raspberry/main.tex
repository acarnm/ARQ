\documentclass[aspectratio=169,11pt]{beamer}

% Tema y paquetes
\usetheme{Madrid}
\usecolortheme{seahorse}
\usefonttheme{professionalfonts}
\setbeamertemplate{navigation symbols}{}
\usepackage[spanish]{babel}
\usepackage[utf8]{inputenc}
\usepackage{graphicx}
\usepackage{hyperref}
\usepackage{listings}
\usepackage{xcolor}
\usepackage{colortbl}
\usepackage{tikz}

% Estilo de código
\lstset{
  basicstyle=\ttfamily\footnotesize,
  keywordstyle=\color{blue},
  stringstyle=\color{green!50!black},
  commentstyle=\color{gray},
  frame=single,
  breaklines=true,
  showstringspaces=false
}

% Datos de la presentación
\title{Introducción y uso práctico de Raspberry Pi}
\subtitle{Conceptos, modelos, usos y programación}
\author{Abel Carnicero\texorpdfstring{\\ \url{acarnm@unileon.es}}{}}
\date{\today}

\begin{document}

% Portada
\begin{frame}
  \titlepage
\end{frame}

% Índice
\begin{frame}{Contenido}
  \tableofcontents
\end{frame}

% ============================================================
\section{¿Qué es Raspberry Pi?}

\begin{frame}[allowframebreaks]{Definición}
  \begin{block}{¿Qué es?}
    Un \textbf{ordenador completo del tamaño de una tarjeta de crédito} (SBC: Single Board Computer), que puede ser conectada a un monitor o una TV, y usarse con un mouse y teclado estándar.
  \end{block}

  \vspace{0.5cm}

  \begin{columns}[T]
    \column{0.5\textwidth}
    \textbf{\color{blue}Características clave:}
    \begin{itemize}
      \item[\checkmark] ordenador completo funcional
      \item[\checkmark] Precio: desde 5€ hasta 100€
      \item[\checkmark] Bajo consumo energético (3-5W)
      \item[\checkmark] Puertos GPIO programables
      \item[\checkmark] Sistema operativo Linux
    \end{itemize}


    \column{0.5\textwidth}
    \textbf{\color{blue}Origen y propósito:}
    \begin{itemize}
      \item Creada en \textbf{2012} por la \textit{Raspberry Pi Foundation} (UK)
      \item Objetivo: \textbf{democratizar la informática}
      \item Más de \textbf{50 millones} de unidades vendidas
      \item Usada en educación, industria y hobbies
    \end{itemize}
  \end{columns}

  \framebreak
  \vspace{0.3cm}
  \begin{alertblock}{¿Por qué es revolucionaria?}
    Convierte la programación y electrónica en algo \textbf{accesible, económico y práctico} para cualquier persona.
  \end{alertblock}
  Ha democratizado el acceso a la tecnología permitiendo que estudiantes, aficionados y profesionales puedan crear desde simples proyectos de aprendizaje hasta \textbf{sistemas industriales complejos}. Su versatilidad la hace ideal tanto para \textbf{enseñar conceptos básicos de programación} a niños como para desarrollar \textbf{prototipos IoT avanzados} en empresas.
\end{frame}

\begin{frame}[allowframebreaks]{Partes principales}
  \begin{columns}
    \column{0.5\textwidth}
    \begin{itemize}
      \item CPU y GPU (procesador ARM)
      \item RAM
      \item Puertos USB y HDMI
      \item Conector GPIO
      \item Conector de cámara (CSI)
      \item Lector microSD
    \end{itemize}
    \column{0.5\textwidth}
    % Imagen de Raspberry Pi - reemplazar con ruta real
    \includegraphics[width=\textwidth]{img/raspy4.jpg}

  \end{columns}

  \framebreak

  \begin{center}
    \includegraphics[width=0.95\textwidth]{img/esquema-conectores-raspberry-pi4.png}
  \end{center}

  \vspace{3cm}

  \framebreak

  \textbf{\color{blue}Componentes explicados:}

  \vspace{0.3cm}


  \begin{itemize}
    \item \textbf{CPU/GPU (SoC):} Procesador ARM Broadcom que ejecuta todas las operaciones. Incluye la GPU para gráficos y video.
    \item \textbf{RAM:} Memoria de acceso aleatorio (1-8GB según modelo) para ejecutar programas y sistema operativo.
    \item \textbf{GPIO (40 pines):} Pines de entrada/salida programables para conectar sensores, LEDs, motores y otros dispositivos electrónicos.
    \item \textbf{Puertos USB:} Para conectar teclado, ratón, memorias USB y otros periféricos. USB 3.0 en Pi 4/5.
  \end{itemize}

  \framebreak

  \begin{itemize}
    \item \textbf{HDMI:} Salida de video y audio digital para conectar monitores o TVs (hasta 4K en Pi 4/5).
    \item \textbf{Puerto Ethernet (RJ45):} Conexión de red cableada (Gigabit en Pi 4/5).
    \item \textbf{Conector microSD:} Almacenamiento principal donde se instala el sistema operativo y archivos.
    \item \textbf{Puerto CSI/DSI:} Para conectar la cámara oficial (CSI) o pantallas táctiles (DSI).
    \item \textbf{USB-C/microUSB:} Alimentación eléctrica (5V, 3A recomendado para Pi 4/5).
  \end{itemize}

\end{frame}

\begin{frame}[allowframebreaks]{Precauciones y buen uso}

  \begin{alertblock}{\textbf{Seguridad física}}
    \begin{itemize}
      \item Manipularla en superficies \textbf{no conductoras} (evitar metal, aluminio).
      \item \textbf{No tocar} los pines GPIO mientras está encendida.
      \item Evitar electricidad estática - descargarse antes de manipular.
      \item Considerar usar una \textbf{carcasa protectora} para evitar cortocircuitos.
    \end{itemize}
  \end{alertblock}

  \vspace{0.3cm}

  \begin{block}{\textbf{Alimentación eléctrica}}
    \begin{itemize}
      \item Usar una fuente de alimentación \textbf{oficial o de calidad} (5V, 3A para Pi 4/5).
      \item Alimentación insuficiente causa \textbf{reinicios aleatorios} y corrupción de datos.
      \item Síntoma común: icono de rayo amarillo en pantalla = \textcolor{red}{voltaje bajo}.
      \item No conectar/desconectar periféricos con la Pi encendida.
    \end{itemize}
  \end{block}

  \framebreak

  \begin{block}{\textbf{Gestión del sistema}}
    \begin{itemize}
      \item \textbf{NUNCA} desconectar la energía sin apagar el sistema operativo.
      \item Usar siempre: \texttt{sudo shutdown -h now} o menú de apagado.
      \item Apagado incorrecto puede \textbf{corromper la tarjeta SD}.
      \item Hacer \textbf{copias de seguridad} regulares de proyectos importantes.
    \end{itemize}
  \end{block}

  \vspace{0.3cm}

  \begin{block}{\textbf{Temperatura y ventilación}}
    \begin{itemize}
      \item Temperatura óptima: \textbf{$<$ 70°C} (revisar con \texttt{vcgencmd measure\_temp}).
      \item Usar \textbf{disipadores de calor} o ventiladores activos en proyectos intensivos.
      \item No cubrir la placa durante operación - necesita \textbf{ventilación}.
      \item El throttling térmico reduce rendimiento automáticamente.
    \end{itemize}
  \end{block}

  \framebreak

  \begin{block}{\textbf{Uso de GPIO - Precauciones críticas}}
    \begin{itemize}
      \item Los pines GPIO son \textbf{3.3V} - NO conectar componentes de 5V directamente.
      \item Corriente máxima por pin: \textbf{16mA} - usar transistores para cargas mayores.
      \item Algunos pines tienen funciones especiales (I2C, SPI, UART) - documentar antes de usar.
      \item \textbf{Conexión incorrecta puede dañar permanentemente} la Raspberry Pi.
    \end{itemize}
  \end{block}

  \vspace{0.3cm}

  \begin{exampleblock}{\textbf{Buenas prácticas recomendadas}}
    \begin{itemize}
      \item Actualizar regularmente: \texttt{sudo apt update \&\& sudo apt upgrade}
      \item Usar tarjetas SD de \textbf{calidad} (Clase 10, A1 o superior).
      \item Documentar conexiones y proyectos (esquemas, fotos, código comentado).
      \item Unirse a la \textbf{comunidad} - foros, Reddit (\texttt{r/raspberry\_pi}).
    \end{itemize}
  \end{exampleblock}

\end{frame}

% ============================================================
\section{Modelos y diferencias}

\begin{frame}[allowframebreaks]{Modelos principales}

  \begin{columns}[c]
    \column{0.4\textwidth}
    \textbf{\color{blue}Raspberry Pi 1 (A, B)}
    \begin{itemize}
      \item Primer modelo (2012)
      \item CPU: 700 MHz ARM11
      \item RAM: 256-512 MB
      \item Revolucionó la educación
    \end{itemize}

    \column{0.6\textwidth}
    \begin{center}
      \includegraphics[width=0.9\textwidth]{img/rp1.jpg}
    \end{center}
  \end{columns}

  \framebreak

  \begin{columns}[c]
    \column{0.4\textwidth}
    \textbf{\color{blue}Raspberry Pi 2}
    \begin{itemize}
      \item Lanzamiento: 2015
      \item CPU: 900 MHz quad-core ARM Cortex-A7
      \item RAM: 1 GB
      \item 6× más rápida que Pi 1
    \end{itemize}

    \column{0.6\textwidth}
    \begin{center}
      \includegraphics[width=0.9\textwidth]{img/rp2.jpg}
    \end{center}
  \end{columns}

  \framebreak

  \begin{columns}[c]
    \column{0.5\textwidth}
    \textbf{\color{blue}Raspberry Pi 3 (B, B+)}
    \begin{itemize}
      \item Lanzamiento: 2016-2018
      \item CPU: 1.2-1.4 GHz quad-core ARM Cortex-A53
      \item RAM: 1 GB
      \item \textbf{Wi-Fi y Bluetooth integrados}
      \item Ideal para proyectos IoT
    \end{itemize}

    \column{0.5\textwidth}
    \begin{center}
      \includegraphics[width=0.9\textwidth]{img/rp3.jpg}
    \end{center}
  \end{columns}

  \vspace{0.5cm}

  \begin{columns}[c]
    \column{0.5\textwidth}
    \textbf{\color{blue}Raspberry Pi 4}
    \begin{itemize}
      \item Lanzamiento: 2019
      \item CPU: 1.5 GHz quad-core ARM Cortex-A72
      \item RAM: 2/4/8 GB LPDDR4
      \item USB 3.0, dual micro-HDMI 4K
      \item Gigabit Ethernet real
    \end{itemize}

    \column{0.5\textwidth}
    \begin{center}
      \includegraphics[width=0.9\textwidth]{img/raspy4.jpg}
    \end{center}
  \end{columns}

  \framebreak

  \begin{columns}[c]
    \column{0.5\textwidth}
    \textbf{\color{blue}Raspberry Pi 5}
    \begin{itemize}
      \item Lanzamiento: 2023
      \item CPU: 2.4 GHz quad-core ARM Cortex-A76
      \item RAM: 4/8 GB LPDDR4X
      \item GPU VideoCore VII mejorada
      \item PCIe 2.0, botón de encendido
      \item 2-3× más rápida que Pi 4
    \end{itemize}

    \column{0.5\textwidth}
    \begin{center}
      \includegraphics[width=0.9\textwidth]{img/rp5.jpg}
    \end{center}
  \end{columns}

  \vspace{0.5cm}

  \begin{columns}[c]
    \column{0.5\textwidth}
    \textbf{\color{blue}Raspberry Pi Zero / Zero 2W}
    \begin{itemize}
      \item Tamaño ultra compacto (65×30 mm)
      \item Zero: 1 GHz single-core (desde \$5)
      \item Zero 2W: 1 GHz quad-core + Wi-Fi/BT
      \item Ideal para proyectos embebidos
      \item Menor consumo energético
    \end{itemize}

    \column{0.5\textwidth}
    \begin{center}
      \includegraphics[width=0.9\textwidth]{img/rp_zero.jpg}
    \end{center}
  \end{columns}

\end{frame}

\begin{frame}[allowframebreaks]{Comparativa rápida}

  \small
  \begin{center}
    \renewcommand{\arraystretch}{1.3}
    \begin{tabular}{|l|c|c|c|c|}
      \hline
      \rowcolor{blue!20}
      \textbf{Característica} & \textbf{Pi 3B+}  & \textbf{Pi 4B}   & \textbf{Pi 5}                                 & \textbf{Zero 2W}                \\
      \hline
      \textbf{CPU}            & 1.4 GHz          & 1.5 GHz          & \cellcolor{green!20}\textbf{2.4 GHz}          & 1.0 GHz                         \\
                              & 4-core           & 4-core           & \cellcolor{green!20}4-core                    & 4-core                          \\
      \hline
      \textbf{RAM}            & 1 GB             & 2-8 GB           & \cellcolor{green!20}\textbf{4-8 GB}           & 512 MB                          \\
      \hline
      \textbf{Wi-Fi/BT}       & $\checkmark$ 4.2 & $\checkmark$ 5.0 & \cellcolor{green!20}\textbf{$\checkmark$ 5.3} & $\checkmark$ 4.2                \\
      \hline
      \textbf{USB}            & 4$\times$ 2.0    & 2$\times$ 3.0    & \cellcolor{green!20}\textbf{2$\times$ 3.0}    & 1$\times$ 2.0                   \\
                              &                  & 2$\times$ 2.0    & \cellcolor{green!20}2$\times$ 2.0             & (OTG)                           \\
      \hline
      \textbf{Video}          & 1$\times$ HDMI   & 2$\times$ micro  & \cellcolor{green!20}\textbf{2$\times$ micro}  & 1$\times$ mini                  \\
                              & 1080p            & \textbf{4K@60}   & \cellcolor{green!20}\textbf{4K@60}            & HDMI                            \\
      \hline
      \textbf{Ethernet}       & \textbf{Gigabit} & \textbf{Gigabit} & \cellcolor{green!20}\textbf{Gigabit}          & \textcolor{red}{$\times$}       \\
      \hline
      \textbf{Precio aprox.}  & 35€              & 45-75€           & \textbf{60-80€}                               & \textcolor{green}{\textbf{15€}} \\
      \hline
    \end{tabular}
    \renewcommand{\arraystretch}{1}
  \end{center}

  \vspace{0.3cm}
  \framebreak

  \begin{exampleblock}{\textbf{¿Cuál elegir?}}
    \small
    \begin{itemize}
      \item \textbf{Educación/básico:} Pi 3B+ o Zero 2W
      \item \textbf{Uso general:} Pi 4B (4GB)
      \item \textbf{Alto rendimiento:} Pi 5 (8GB)
      \item \textbf{Embebido:} Zero 2W
    \end{itemize}
  \end{exampleblock}


  \begin{alertblock}{\textbf{Rendimiento relativo}}
    \small
    \begin{itemize}
      \item Pi 3 $\rightarrow$ Pi 4: \textbf{+100\%}
      \item Pi 4 $\rightarrow$ Pi 5: \textbf{+200\%}
      \item Zero 2W $\approx$ Pi 3 en CPU
    \end{itemize}
  \end{alertblock}

\end{frame}

% ============================================================
\section{Qué se puede hacer con Raspberry Pi}

\begin{frame}{Usos comunes}
  \begin{columns}[T]
    \column{0.48\textwidth}
    \begin{block}{\textbf{Entretenimiento y Multimedia}}
      \begin{itemize}
        \item[\textcolor{orange}{$\blacktriangleright$}] Centro multimedia (Kodi, Plex)
        \item[\textcolor{orange}{$\blacktriangleright$}] Retroconsola (RetroPie)
        \item[\textcolor{orange}{$\blacktriangleright$}] Consola de juegos retro
      \end{itemize}
    \end{block}

    \vspace{0.3cm}

    \begin{block}{\textbf{Servidor y Red}}
      \begin{itemize}
        \item[\textcolor{green!60!black}{$\blacktriangleright$}] Servidor web/archivos
        \item[\textcolor{green!60!black}{$\blacktriangleright$}] Servidor de impresión
        \item[\textcolor{green!60!black}{$\blacktriangleright$}] NAS (almacenamiento)
      \end{itemize}
    \end{block}

    \column{0.48\textwidth}
    \begin{block}{\textbf{IoT y Automatización}}
      \begin{itemize}
        \item[\textcolor{cyan}{$\blacktriangleright$}] Domótica e IoT
        \item[\textcolor{cyan}{$\blacktriangleright$}] Estación meteorológica
        \item[\textcolor{cyan}{$\blacktriangleright$}] Cámaras de seguridad
      \end{itemize}
    \end{block}

    \vspace{0.3cm}

    \begin{block}{\textbf{Educación y Desarrollo}}
      \begin{itemize}
        \item[\textcolor{red!70!black}{$\blacktriangleright$}] Robótica y control
        \item[\textcolor{red!70!black}{$\blacktriangleright$}] Aprendizaje de programación
        \item[\textcolor{red!70!black}{$\blacktriangleright$}] Prototipado electrónico
      \end{itemize}
    \end{block}
  \end{columns}
\end{frame}

\begin{frame}[allowframebreaks]{Proyectos educativos}

  \begin{columns}[T]
    \column{0.48\textwidth}
    \begin{block}{\textbf{Programación}}
      \textbf{\color{orange}Python:}
      \begin{itemize}
        \item Control de hardware con GPIO Zero
        \item Automatización de tareas
        \item Análisis de datos con pandas
      \end{itemize}
      \vspace{0.2cm}
      \textbf{\color{orange}Scratch:}
      \begin{itemize}
        \item Programación visual para niños
        \item Control de LEDs y sensores
        \item Primeros pasos en lógica
      \end{itemize}
    \end{block}

    \column{0.48\textwidth}
    \begin{block}{\textbf{Electrónica Básica}}
      \textbf{\color{green!60!black}Hardware:}
      \begin{itemize}
        \item LEDs: control de brillo (PWM)
        \item Sensores: temperatura, distancia, luz
        \item Servomotores: control de posición
        \item Cámara Pi: visión artificial
      \end{itemize}
      \vspace{0.2cm}
      \textbf{\color{green!60!black}Conceptos:}
      \begin{itemize}
        \item Circuitos digitales (GPIO)
        \item Señales PWM y analógicas
        \item Protocolos: I2C, SPI, UART
      \end{itemize}
    \end{block}
  \end{columns}

  \framebreak

  \begin{exampleblock}{\textbf{Ejemplo 1: Sistema de alarma}}
    \small
    \textbf{Descripción:}
    \begin{itemize}
      \item Alarma que detecta movimiento con sensor PIR y activa un buzzer y LED RGB.
    \end{itemize}
    \begin{columns}
      \column{0.48\textwidth}
      \textbf{Componentes:}
      \begin{itemize}
        \item Sensor PIR (movimiento)
        \item Buzzer
        \item LED RGB
        \item Python + GPIO Zero
      \end{itemize}

      \column{0.48\textwidth}
      \textbf{Aprendizaje:}
      \begin{itemize}
        \item Entrada digital (sensor)
        \item Salida digital (LED/buzzer)
        \item Lógica condicional
        \item Manejo de eventos
      \end{itemize}

    \end{columns}

    \vspace{0.2cm}

  \end{exampleblock}

  \framebreak

  \begin{exampleblock}{\textbf{Ejemplo 2: Robot evita obstáculos}}
    \small
    \textbf{Descripción:}
    \begin{itemize}
      \item Robot móvil que usa sensor ultrasónico para detectar obstáculos y cambiar de dirección automáticamente.
    \end{itemize}

    \begin{columns}
      \column{0.48\textwidth}
      \textbf{Componentes:}
      \begin{itemize}
        \item Sensor ultrasónico HC-SR04
        \item Motores DC + driver L298N
        \item Chasis con ruedas
        \item Python para control
      \end{itemize}
      \vspace{0.2cm}
      \column{0.48\textwidth}
      \textbf{Aprendizaje:}
      \begin{itemize}
        \item Medición de distancia
        \item Control de motores (PWM)
        \item Algoritmos de decisión
        \item Integración de sistemas
      \end{itemize}
    \end{columns}
  \end{exampleblock}

  \framebreak

  \begin{block}{\textbf{Linux y Redes}}
    \begin{columns}
      \column{0.48\textwidth}
      \textbf{\color{purple}Administración:}
      \begin{itemize}
        \item Línea de comandos (bash)
        \item Gestión de archivos y permisos
        \item Instalación de software (apt)
        \item Scripts de automatización
      \end{itemize}
      \vspace{0.2cm}
      \column{0.48\textwidth}
      \textbf{\color{purple}Networking:}
      \begin{itemize}
        \item Configuración de red (Wi-Fi/Ethernet)
        \item Servidor web (Apache/Nginx)
        \item SSH y acceso remoto
        \item Protocolos: HTTP, MQTT, TCP/IP
      \end{itemize}
    \end{columns}

  \end{block}


\end{frame}

% ============================================================
\section{Instalación y configuración}

\begin{frame}{Sistema operativo}
  \begin{itemize}
    \item Usar \textbf{Raspberry Pi Imager} $\rightarrow$ \url{https://www.raspberrypi.com/software/}.
    \item Sistema recomendado: \textbf{Raspberry Pi OS (basado en Debian)}.
    \item Alternativas: Ubuntu, RetroPie, LibreELEC, etc.
  \end{itemize}
  \vspace{0.5cm}
  \centering
  \includegraphics[width=0.9\textheight]{img/rp_imager.png}
\end{frame}

% ============================================================
\section{Uso básico y software}

\begin{frame}{Software recomendado}
  \begin{columns}[T]
    \column{0.48\textwidth}
    \begin{block}{\textbf{Desarrollo}}
      \begin{itemize}
        \item[\textcolor{orange}{$\star$}] \textbf{VS Code} - Editor profesional
        \item[\textcolor{orange}{$\star$}] \textbf{Thonny} - IDE para principiantes
      \end{itemize}
    \end{block}

    \column{0.48\textwidth}
    \begin{block}{\textbf{Programación}}
      \begin{itemize}
        \item[\textcolor{green!60!black}{$\diamond$}] \textbf{Python 3} - Lenguaje principal
        \item[\textcolor{green!60!black}{$\diamond$}] \textbf{GPIO Zero} - Control hardware
      \end{itemize}
    \end{block}
  \end{columns}

  \vspace{0.5cm}

  \begin{exampleblock}{\centering \textbf{Stack tecnológico recomendado}}
    \centering
    \large
    \textcolor{blue}{\textbf{Thonny}} + \textcolor{orange}{\textbf{Python 3}} + \textcolor{purple}{\textbf{GPIO Zero}} = \textcolor{green!60!black}{\textbf{¡Éxito!}}
  \end{exampleblock}
\end{frame}

% ============================================================
\section{Programación y ejemplos}

\begin{frame}[fragile]{Ejemplo básico en Python}
  \begin{lstlisting}[language=Python]
from gpiozero import LED
from time import sleep

led = LED(17)

while True:
    led.on()
    sleep(1)
    led.off()
    sleep(1)
  \end{lstlisting}
  \textbf{Ejemplo:} parpadeo de LED conectado al pin GPIO17.
\end{frame}

\begin{frame}[fragile]{Lectura de sensor}
  \begin{lstlisting}[language=Python]
from gpiozero import MotionSensor

pir = MotionSensor(4)

while True:
    pir.wait_for_motion()
    print("Movimiento detectado!")
    pir.wait_for_no_motion()
  \end{lstlisting}
\end{frame}


\end{document}
